\documentclass[onecolumn,11pt]{IEEEtran}
\newcommand{\subparagraph}{}
\usepackage{titlesec}

\title{Project Title}
\author{\textbf{Project Leader (email@domain.com)}, Project Associates (email@domain.com)}

\titleformat{\section}
  {\centering\Large\bf}{\thesection}{0.2em}{}
\titlespacing*{\section}{0em}{0.5em}{0.5em}


\begin{document}
\maketitle

\section*{Motivation}

Describe your project's motivation, which problems need to be addressed, and/or which coding tasks need to be accomplished. Please be concise and specific.

\section*{Contribution to NDN}

Describe how the project will benefit Named Data Networking architecture. For example, a new application that demonstrates advantages of NDN; a new code that can be used by other NDN apps developers; a new NDN tool to simplify operational tasks; an adaptation of NDN to new languages/new platforms.

\section*{Tasks}

Given only 12 hours are available for actual hacking during the Hackathon, please pre-plan specific hacking tasks. This will help you organize the hacking team and others can  choose tasks according to their expertise.

\section*{Required Knowledge for Participants}

List language(s), libraries and any other tools that the participants should be familiar with.

\section*{Expected Outcome}

List the expected results of the Hackathon. Depending on the project, it can be fully or partially accomplished during the Hackathon. Include information about what you expect to be demonstrated to the judges at the end of the Hackathon.

\end{document}
